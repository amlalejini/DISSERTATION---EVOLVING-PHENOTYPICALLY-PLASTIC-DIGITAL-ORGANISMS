
% The intro has long wind-up time, this will be direct and to the point.

% @AML: Really want this to be MUCH MUCH shorter. Maybe don't go into all the details? Just keep it at a high level?
The ability to dynamically respond to exogenous cues from the environment is a fundamental feature of complex adaptive systems.
In biology, such responsiveness is called phenotypic plasticity, which is the capacity for a single genotype to express different phenotypes in response to a change in its environment.
% In this dissertation, I use digital evolution experiments to explore the process by which adaptive plasticity evolves and to illuminate its effects on subsequent evolutionary dynamics.   
Biologists have long been interested in understanding how adaptive phenotypic plasticity evolves, the mechanisms underpinning plasticity in natural organisms, and how the evolution of plasticity influences subsequent evolutionary outcomes.
Yet, few empirical studies using biological model organisms have been able to observe both the initial evolution of adaptive phenotypic plasticity and its subsequent evolutionary outcomes.
In this dissertation, I use populations of self-replicating computer programs (digital organisms) to empirically study the conditions under which adaptive phenotypic plasticity evolves and how its evolution shapes subsequent evolutionary outcomes.
I show that both mutation rate and environmental change rate influence the potential for adaptive plasticity to evolve, and I show that both unconditional trait expression and imprecise forms of plasticity are important evolutionary building blocks for the evolution of adaptive plasticity. 
Next, I investigate how the \textit{de novo} evolution of adaptive plasticity influences subsequent evolutionary change and the evolution and maintenance of novel adaptive traits.
I find that adaptive phenotypic plasticity stabilizes populations against environmental changes, which allows populations to more easily retain novel adaptive traits. 

Phenotypic plasticity also has practical applications in evolutionary computing wherein we exploit the natural principles of evolution as a general purpose search algorithm to solve challenging computational problems. 
As in biological organisms, phenotypic plasticity can allow evolved solutions to be robust to noise and capable of dynamically responding to changing problem conditions.
I focus on genetic programming wherein we apply evolutionary algorithms to automatically synthesize computer programs rather than writing them by hand.
Specifically, I draw on our understanding of biological mechanisms of adaptive plasticity and their evolution to develop novel techniques for evolving more dynamically responsive computer programs.
First, I introduce SignalGP, a novel genetic programming technique for evolving event-driven computer programs, and I show that SignalGP allows us to evolve programs better able to rapidly interact with the environment or with other agents. 
Next, I demonstrate tag-based genetic regulation, a new genetic programming technique that allows programs to dynamically adjust the code modules that they execute, and I show that tag-based regulation improves problem-solving performance on context-dependent problems where programs must adjust how they respond to current inputs based on prior inputs.
Finally, I briefly introduce tag-accessed memory, a novel genetic programming technique that provides a more flexible mechanism for programs to define and use evolvable variable names.