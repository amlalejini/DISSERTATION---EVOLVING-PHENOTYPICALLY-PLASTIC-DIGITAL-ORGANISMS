% --- Abstract ---

The ability to dynamically respond to cues from the environment is a fundamental requirement of any adaptive system.
In biological systems, changes to an organism based on environmental cues is called phenotypic plasticity.
Indeed, phenotypic plasticity underlies many of the adaptive traits and developmental patterns found in nature and serves as a key mechanism for responding to spatially or temporally variable environments.
Most computer programs require phenotypic plasticity, as they must respond dynamically to stimuli such as user input, sensor data, \textit{et cetera}. 
As such, phenotypic plasticity also has practical applications in genetic programming, wherein we apply the natural principles of evolution to automatically synthesize computer programs rather than writing them by hand.

In this dissertation, I achieve two synergistic aims: (1) I use populations of self-replicating computer programs (digital organisms) to empirically study the conditions under which adaptive phenotypic plasticity evolves and how its evolution shapes subsequent evolutionary outcomes; and, (2) I transfer insights from biology to develop novel genetic programming techniques in order to evolve more responsive (\textit{i.e.}, phenotypically plastic) computer programs. % than allowed by existing state of the art...
First, I illustrate the importance of mutation rate, environmental change, and partially-plastic building blocks on the evolution of adaptive plasticity. 
Next, I show that adaptive phenotypic plasticity stabilizes populations against environmental change, allowing them to more easily retain novel adaptive traits. 
Finally, I improve our ability to evolve phenotypically plastic computer programs with three novel genetic programming techniques: 
(1) SignalGP, which provides mechanisms to control code expression based on environmental cues, (2) tag-based genetic regulation to adjust code expression based on current context, and (3) tag-accessed memory to provide more dynamic machnisms for storing data.



