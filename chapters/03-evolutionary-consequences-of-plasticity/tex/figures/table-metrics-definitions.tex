%%%%%%%%%%%%%%%%%%% METRIC DEFINITIONS (used in table) %%%%%%%%%%%%%%%%%%%
\newcommand{\SweepsMetricName}{
Coalescence event count
}
\newcommand{\SweepsMetricDesc}{
Number of coalescence events that have occurred, which indicates the frequency of selective sweeps in the population.
}

\newcommand{\MutationCountMetricName}{
Mutation count
}
\newcommand{\MutationCountMetricDesc}{
Sum of all mutations that have occurred along a lineage.
}

\newcommand{\PhenotypicVolatilityMetricName}{
Phenotypic volatility
}
\newcommand{\PhenotypicVolatilityMetricDesc}{
Number of instances where parent and offspring phenotypic profiles do not match along a lineage.
Phenotypic volatility as defined here indicates the rate at which accumulated genetic changes actually change the phenotype along a lineage.
}

\newcommand{\MutationalStabilityMetricName}{
Mutational stability
}
\newcommand{\MutationalStabilityMetricDesc}{
Proportion of mutated offspring along a lineage whose phenotypic profile matches that of their parent. 
}

\newcommand{\GenotypicFidelityMetricName}{
Genotypic fidelity
}
\newcommand{\GenotypicFidelityMetricDesc}{
Frequency that an offspring's genotype is identical to its parent's genotype along a given lineage.
}

\newcommand{\PhenotypicFidelityMetricName}{
Phenotypic fidelity
}
\newcommand{\PhenotypicFidelityMetricDesc}{
Frequency that an offspring's phenotypic profile is identical to its parent's phenotypic profile along a given lineage.
}

\newcommand{\TaskPerformanceMetricName}{
Final novel task count
}
\newcommand{\TaskPerformanceMetricDesc}{
Count of unique novel tasks performed by the representative organism in a final population from experiment \hyperref[chapter:consequences-of-plasticity:sec:methods:experiment:novel-task-evolution]{phase 2B}. 
This metric can range from 0 to 71 and measures the level of exploitation of the fitness landscape (\textit{i.e.}, the mapping between genetic space and phenotype space) at a given point in time.
}

\newcommand{\TaskDiscoveryMetricName}{
Novel task discovery
}
\newcommand{\TaskDiscoveryMetricDesc}{
Number of unique novel tasks ever performed along a given lineage in experimental \hyperref[chapter:consequences-of-plasticity:sec:methods:experiment:novel-task-evolution]{phase 2B}, even if a task is later lost.
This metric can range from 0 to 71 and measures a given lineage's level of exploration of the fitness landscape.
}

\newcommand{\TaskLossMetricName}{
Novel task loss
}
\newcommand{\TaskLossMetricDesc}{
Number of instances along a given lineage from experimental \hyperref[chapter:consequences-of-plasticity:sec:methods:experiment:novel-task-evolution]{phase 2B} where a novel task is performed by a parent but not its offspring. 
This metric measures how often a given lineage fails to retain evolved traits over time.
}

\newcommand{\FinalPoisonMetricName}{
Final poisonous task count
}
\newcommand{\FinalPoisonMetricDesc}{
Number of times the poisonous task is performed by the representative organism from a final population from experiment \hyperref[chapter:consequences-of-plasticity:sec:methods:experiment:deleterious-instruction-accumulation]{phase 2C}.
}

\newcommand{\LineagePoisonMetricName}{
Poisonous task acquisition count
}
\newcommand{\LineagePoisonMetricDesc}{
Number of instances along a given lineage where a mutation causes an offspring perform the poisonous task more times than its parent. 
}

\newcommand{\ArchitectureVolatilityMetricName}{
Architectural volatility
}
\newcommand{\ArchitectureVolatilityMetricDesc}{
The average number of loci in the genome that change function per mutation along a lineage. 
%Loci function is denoted as the combination of task machinery, plasticity machinery, vestigial machinery for both ENV-A and ENV-B tasks, as well as replication and required machinery. 
}


%%%%%%%%%%%%%%%%%%% TABLE %%%%%%%%%%%%%%%%%%%

\newcommand*{\thead}[1]{\multicolumn{1}{c}{\bfseries #1}}

\setlength{\tabcolsep}{12pt}
\renewcommand{\arraystretch}{1.1}
\begin{table}[ht!]
    \centering
    
    \rowcolors{2}{gray!25}{white}
    \begin{tabularx}{\linewidth}{lX} % p{10cm}
        \rowcolor{gray!50}
        \hline 
        \thead{Metric} & \thead{Description}   \\
        \hline
        \SweepsMetricName & \SweepsMetricDesc \\
        \MutationCountMetricName & \MutationCountMetricDesc \\
        \PhenotypicVolatilityMetricName & \PhenotypicVolatilityMetricDesc \\
        \MutationalStabilityMetricName & \MutationalStabilityMetricDesc \\
        \ArchitectureVolatilityMetricName & \ArchitectureVolatilityMetricDesc \\
        \TaskPerformanceMetricName & \TaskPerformanceMetricDesc \\
        \TaskDiscoveryMetricName & \TaskDiscoveryMetricDesc \\
        \TaskLossMetricName & \TaskLossMetricDesc \\
        \FinalPoisonMetricName & \FinalPoisonMetricDesc \\
        \LineagePoisonMetricName & \LineagePoisonMetricDesc \\
        \hline
    \end{tabularx}
    
    \caption{\small \textbf{Metric descriptions.}}
    \label{chapter:consequences-of-plasticity:tab:metrics-definitions}
\end{table}
\renewcommand{\arraystretch}{1.0}


