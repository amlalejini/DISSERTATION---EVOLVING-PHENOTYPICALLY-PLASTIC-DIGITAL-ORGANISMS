\chapter{Conclusions}
\label{chapter:conclusions}
%%%%%%%%%%%%%%%%%%%%%%%%%%%%% NOTES %%%%%%%%%%%%%%%%%%%%%%%%%%%%%
% All natural environments subject populations to some form of change.

% ; our findings suggest that the stabilizing effect of phenotypic plasticity plays an important role in subsequent adaptive evolution.

% mention DISHTINY
% mention lexicase 
% AAGOS understanding changing environments
%%%%%%%%%%%%%%%%%%%%%%%%%%%%% NOTES %%%%%%%%%%%%%%%%%%%%%%%%%%%%%

% TODO

Dissertation is split between two synergistic focuses: [blah] and [blah].
[Riff on Thesis statement].
[How I have achieved thesis statement].

\section{Contributions}

In summary, this dissertation makes the following contributions:

\begin{itemize}
    % --- Origins of plasticity ---
    \item In \textbf{Chapter \ref{chapter:evolutionary-origins-of-phenotypic-plasticity}}, [I/We?] found that both environmental change rate and mutation rate influence the likelihood for adaptive phenotypic plasticity to evolve in populations of digital organisms. 
    By analyzing the lineages of plastic organisms, I identified unconditional trait expression and imperfect forms of phenotypic plasticity as important evolutionary building blocks for the evolution of adaptive plasticity. 
    
    % --- Consequences of plasticity ---
    \item In \textbf{Chapter \ref{chapter:evolutionary-consequences-of-plasticity}}, I used populations of digital organisms to empirically test whether the evolution of adaptive phenotypic plasticity alters evolutionary dynamics and influences evolutionary outcomes in cyclically changing environments.
    I found that the evolution of adaptive phenotypic plasticity stabilizes populations against environmental changes and constrains the rate of subsequent evolutionary change.
    By buffering populations against environmental change, adaptive plasticity improved novel trait retention and reduced the accumulation of deleterious mutations relative to non-plastic populations evolved in an otherwise identical environment. 
    
    % --- SignalGP ---
    \item In \textbf{Chapter \ref{chapter:signalgp}}, I introduced SignalGP, a novel genetic programming technique for evolving event-driven computer programs. 
    I showed that SignalGP allows us to evolve programs better able to rapidly interact the environment or with other agents. 
    % Further, SignalGP model digital organism. 
    
    % --- Tag-based genetic regulation ---
    \item In \textbf{Chapter \ref{chapter:tag-based-regulation}}, I developed tag-based genetic regulation, a new genetic programming technique that allows programs to dynamically adjust which code modules to express.
    I described how to augment existing genetic programming systems with tag-based regulation, and I showed that tag-based regulation improves problem-solving performance on context-dependent problems where programs must adjust how they respond to current inputs based on prior inputs.
    
    % --- Tag-accessed memory ---
    \item In \textbf{Chapter \ref{chapter:tag-accessed-memory}}, I proposed tag-accessed memory, a new mechanism for labeling and identifying memory positions in genetic programming.  
    With preliminary experiments, I found that, under favorable mutation rates, both tag-accessed memory and conventional direct-indexed memory achieve similar performance on a range of program synthesis problems. 
    % [Promising technique for future research].
    
\end{itemize}

\section{Future Directions}

% TODO

Thus far, I have...

\subsection{Broadened applications of SignalGP}

% [Origins of SignalGP].
% - Cell biology class, learning about signal transduction.
% - Working on a project exploring different levels of complexity in plasticity using Avida, but having trouble evolving program capable of effectively independently regulating many tasks in environment. 
% - ROS (event-driven computing).
% - But how 
% - Use bit string signatures to match 
% - Tag-based referencing.

% I have shown it to be effective at X/Y/Z. 
% + added genetic regulation.

% Several next steps:

\subsubsection{Multi-representation SignalGP}
% \label{chapter:conclusion:sec:future-work:multi-representation-signalgp}

In Chapters \ref{chapter:signalgp} and \ref{chapter:tag-based-regulation}, I have exclusively used SignalGP in the context of linear GP.
That is, SignalGP functions (modules) associate a tag with a linear sequence of instructions. 
However, in principle, SignalGP is generalizable across a variety of evolutionary computation representations. 

SignalGP programs are composed of a set of functions where each function is referred to via its tag. 
We can imagine these functions to be black-box input-output machines: when called or triggered by an event, a function is run with input and can produce output by manipulating memory or by generating signals. 
Instead of constructing functions with linear sequences of instructions, I could used other computational substrates (representations) capable of receiving input and producing output (\textit{e.g.}, other GP representations, artificial neural networks, Markov Brains [cite - Hintze2017], hard-coded modules, \textit{etc.}). 
We could even employ a variety of representations within a single SignalGP agent. 

SignalGP's tag-based naming scheme allows us to use this black-box metaphor. 
Functions composed of different representations can still refer to one another via tags, and events are agnostic to the underlying representation used to handle them, requiring only that the representation is capable of processing event-specific data. 
Allowing for such multi-representation agents may complicate the SignalGP virtual hardware, program evaluation, and mutation operators, but in exchange, it would provide evolution with a toolbox of diverse representations. 

Hintze \textit{et al.} proposed and demonstrated the evolutionary ``buffet method'' where Markov brains could be composed of heterogeneous computational substrates, allowing evolution to work out the most appropriate representation for a given problem \citep{hintze_buffet_2019}. 
Indeed, Hintze \textit{et al.} observed that different problems produced solutions with different distributions of component types, making buffet-style Markov brains a flexible representation for solving a range of different types of problems. 
Multi-representation SignalGP provides an unexplored, alternative approach to evolving multi-representation agents, bringing the buffet method into an event-driven context.

\subsubsection{Major Transitions in SignalGP}

In a major evolutionary transition in individuality, formerly distinct individuals unite to form a new, more complex lifeform, redefining what it means to be an individual. 
The evolution of eukaryotes, multi-cellular life, and eusocial insect colonies are all examples of transitions in individuality. 
Often the individuals that make up the higher-level entity are limited to local information, lacking direct access to the global state of the higher-level unit; lower-level units must rely on signaling and sensory information to coordinate their roles in the group [cite - Smith1997,West2015].
In a computational sense, a major transition in individuality is the evolution of a distributed system.
Capturing these types of transitions in GP would give evolution a mechanism to incrementally form distributed systems from formerly individual programs. 

Above, I described how SignalGP could be extended to allow multi-representation programs where functions can be of any representation capable of receiving input and producing output. 
We can take this approach one step further: any module within a SignalGP agent could be \textit{another} (formerly individual) SignalGP agent. 
% This approach is conceptually similar to Tangled Program Graph representation \citep{Kelly2017}.
 
% \begin{figure*}
% 	\centering
%   \includegraphics[width=0.5\textwidth]{sgp-trans.pdf}
%   \caption{\small \textbf{Example of a multi-level SignalGP program.} In this example, the agent is composed of five modules, including a neural network, a Markov Brain, a linear GP representation, and two multi-module programs at a lower level of organization.}
%   \label{fig:major_trans_cartoon}
% \end{figure*}

For example, we can imagine a mutation operator that, when applied, induces transitions in individuality by injecting co-evolving SignalGP programs as self-contained, tagged modules into a mutated program, allowing single individuals to be aggregates of lower-level individuals. 
Further, transitions in individuality can be applied \textit{hierarchically}. 
Biological evolution has examples of such hierarchical transitions: eusocial insect colonies are composed of many multicellular individuals, each of which are composed of many eukaryotic cells, which in turn are composed of organelles (many of which are thought to have been formally distinct individuals).
An individual SignalGP program may be composed of many SignalGP program modules, which may themselves be composed of many SignalGP programs. 

Implementing a mutation operator capable of inducing arbitrary numbers of hierarchical transitions in individuality requires us to answer the following questions: 
How should formerly individual programs interact when forced into an aggregate? 
And, how should an evolutionary algorithm handle evaluating both individuals and aggregates of individuals? 

From the evolutionary algorithm's perspective, a multi-level SignalGP program is indistinguishable from a single-level. 
However, just as biological organisms composed of lower-level units of individuality require more energy to subsist, multi-level SignalGP programs require many more CPU cycles than single-level SignalGP programs. 
This is consistent with biology where major transitions disproportionately occur in energy-rich environments [cite - Smith1997]. 

Extending SignalGP to support hierarchical transitions in individuality could also provide a useful model for studying biological evolutionary transitions, allowing us to ask general questions about their dynamics. 
A transition in individuality mutation operator would also allow us to solve problems that might be best solved by a distributed system without knowing the optimal configuration of that distributed system \textit{a priori}. 

\subsubsection{SignalGP as a new model digital organism}

% TODO

\subsection{The evolutionary origins and consequences of phenotypic plasticity}


 
% - Signal-driven genetic programs as a model digital organism. 

\subsection{Applying evolutionary algorithms to microbial populations}

