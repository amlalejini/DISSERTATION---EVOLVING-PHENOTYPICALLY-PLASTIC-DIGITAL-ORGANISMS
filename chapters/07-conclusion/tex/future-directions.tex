\section{Future Directions}

Thus far, I have focused on using digital evolution techniques to study general principles about evolutionary processes (Chapters \ref{chapter:evolutionary-origins-of-plasticity} and \ref{chapter:evolutionary-consequences-of-plasticity}) and applying inspiration from biology to evolutionary computing (Chapters \ref{chapter:signalgp}, \ref{chapter:tag-based-regulation}, and \ref{chapter:tag-accessed-memory}).
There are many future directions with which to take my research.
Here, I highlight just two (of many) planned directions: broadened applications of SignalGP and transferring techniques and insights from evolutionary computing back into laboratory-based experimental evolution.%[bridging digital and laboratory experimental evolution].

% There are many possible future directions to take my research.
% Below, I outline two broadened applications of SignalGP (introduced in Chapter \ref{chapter:signalgp}), and I discuss my immediate next steps to [bridge digital and laboratory experimental evolution]. 
% @AML: Not sure what balance to strike with the second direction. 

\subsection{Broadened applications of SignalGP}

There are many extensions and applications of SignalGP that I hope to either pursue, facilitate, or eagerly watch others carry out.
Given that SignalGP is a new genetic programming representation with the capacity to address new type of problems, there are many fundamental topics ripe for exploration.
%[basic explorations still to carry out].
For example, I have yet to investigate the effects of crossover or any form of horizontal gene transfer in SignalGP.
In addition, most of the program synthesis problems that I have applied SignalGP to have been primarily diagnostic; in the future, I look forward to more broadly benchmarking SignalGP on challenging event-driven program synthesis problems.  

Below, I discuss in detail two additional extensions to my work: positioning SignalGP as a model organism for digital experimental evolution and a developing a multi-representation version of SignalGP that can bring together the most effect aspects of other GP representations.

\subsubsection{SignalGP as model organism for digital experimental evolution}

In Chapters \ref{chapter:signalgp} and \ref{chapter:tag-based-regulation}, I demonstrated SignalGP in an applied genetic programming context. 
However, one of my original motivations for developing SignalGP was to design a new approach for representing self-replicating computer programs that emphasizes dynamic interactions among digital organisms and between digital organisms and their environment.
In Chapters \ref{chapter:evolutionary-origins-of-plasticity} and \ref{chapter:evolutionary-consequences-of-plasticity}, I experimentally evolved relatively simple forms of adaptive phenotypic plasticity; that is, to achieve an optimal form of plasticity, digital organisms needed to toggle between two relatively simple phenotypes based on environmental conditions. 
In some of my earliest (unpublished) experiments using Avida, I attempted to evolve more complex forms of adaptive plasticity (e.g., the capacity to independently regulate many traits). 
Such adaptive plasticity proved to be challenging to evolve (and thus study) in Avida because genomes are expressed procedurally: actions are performed one at a time in a single chain of execution and must explicitly check for new sensory information.
As such, organisms in Avida must continuously generate explicit queries in order to identify (and react to) any changes in their environment.
Further, the genetic mechanisms in Avida for encoding adaptive plasticity are not easily scalable; I found that plastic programs typically regulate a few key instructions to alter their phenotype and cannot as easily toggle large sequences of instructions on or off.

There are many different model organisms used in biological research, each with their own benefits and shortcomings for conducting evolution experiments. 
Yet, historically, there have been very few different forms of self-replicating computer programs used in digital evolution experiments. 
I envision SignalGP as providing a useful representation for experiments where digital organisms need to dynamically react to signals from the environment or from other agents. 
I look forward to expanding on my work presented in Chapter \ref{chapter:tag-based-regulation} on tag-based regulation to further enhance gene regulation and epigenetic inheritance in SignalGP.   
% I plan to expand on my research from Chapters \ref{chapter:evolutionary-origins-of-plasticity} and \ref{chapter:evolutionary-consequences-of-plasticity} on the evolutionary origins and consequences of adaptive phenotypic plasticity using SignalGP digital organisms. 

% @AML: More specifics here??

I am most excited, however, by ongoing digital evolution work using SignalGP that is being conducted by \textit{other} researchers. 
For example, Matthew Andres Moreno has incorporated SignalGP into the DISHTINY digital evolution platform \citep{moreno_toward_2019,Lalejini_Moreno_Ofria_DISHTINY_2020}.
DISHTINY provides independently replicating digital organisms (cells) with the ability and the incentive to unite into higher-level individuals.
The system demonstrates \textit{de novo} major evolutionary transitions in individuality without direct interventions by the experimenter. 
Individual cells are SignalGP agents, which allows them to respond to the environment and communicate with one another in a signal-driven context.
Moreno has developed novel approaches to SignalGP module regulation and execution \citep{Moreno_2021}, distributed agent-agent interactions \citep{Moreno_Ofria_2020}, and even implemented substantially more efficient versions of the SignalGP virtual hardware \citep{Moreno_Rodriguez-Papa_2021}. 

\subsubsection{Multi-representation SignalGP}

In Chapters \ref{chapter:signalgp} and \ref{chapter:tag-based-regulation}, SignalGP functions (modules) associate a tag with the start of a linear sequence of instructions. 
We can imagine these functions to be black-box input-output machines: when called or triggered by an event, a function is provided with input and can return output, writing to memory or generating signals as it goes. 
Instead of constructing functions with linear sequences of instructions, we could use other computational substrates capable of receiving input and producing output (\textit{e.g.}, other GP representations, artificial neural networks, Markov brains, hard-coded modules, \textit{etc.}). 
We could even employ a variety of representations within a single SignalGP agent. 

SignalGP's tag-based naming scheme enables this black-box metaphor. 
Functions composed of different representations can still refer to one another via tags, and events are agnostic to the underlying representation used to handle them, requiring only that the representation is capable of processing event-specific data. 
Allowing for such multi-representation agents may complicate the SignalGP virtual hardware, program evaluation, and mutation operators, but in exchange, it would provide evolution with a toolbox of diverse representations. 

Hintze et al. proposed and demonstrated the evolutionary ``buffet method'' where Markov brains could be composed of heterogeneous computational substrates, allowing evolution to work out the most appropriate representation for a given problem \citep{hintze_buffet_2019}. 
Indeed, Hintze et al. observed that different problems produced solutions with different distributions of component types, making buffet-style Markov brains a flexible representation for solving a range of different types of problems. 
Multi-representation SignalGP provides an unexplored, alternative approach to evolving multi-representation agents, bringing the buffet method into an event-driven context.

\subsection{Transferring algorithms from evolutionary computing to laboratory-based experimental evolution}
% Alternative titles:
% - Applying evolutionary algorithms to direct the evolution of microbial populations
% - applying evolutionary algorithms to direct the evolution of microbial populations
% - ???

% @AML: need to shape this into a better dissertation section. Currently does not flow well. 

My eventual goal is to work, teach, and mentor seamlessly across computational and laboratory evolution systems, cyclically transferring insights from biology to evolutionary computing and back again.
However, my research thus far has been entirely computational in nature, and I have yet to work with laboratory experimental evolution systems.
To this end, my immediate future plans are to learn more about microbial experimental evolution as a postdoctoral researcher in Dr. Luis Zaman's laboratory at the University of Michigan.

In collaboration with Dr. Zaman's lab, I will use cutting-edge laboratory automation and equipment fabrication technology to conduct evolution experiments that take advantage of precise environmental controls and high-resolution sensor feedback.
One of the initial ways we aim to bridge computational and laboratory experimental evolution is by converting modern evolutionary computing algorithms to direct the evolution of microbial populations. 

Directed evolution wields artificial selection as a tool to generate biomolecules and organisms with enhanced or novel functional traits [cite - Sanchez, Chen, Frances Arnold]. 
The scale and specificity of artificial selection has been revolutionized by a deeper understanding of evolutionary and molecular biology in combination with technological innovations in sequencing, sensing, and laboratory automation.
These advances have cultivated growing interest in directing the evolution of whole microbial communities with functions that can be harnessed in medicine, biotechnology, and agriculture [cite - Sanchez]. 
Yet, the procedures for selecting which communities to propagate tend to focus on those that are the most fit, leaving more nuanced techniques unexplored, despite many highly effective results in evolutionary computation.

%As made clear in this dissertation, directed evolution has not been limited to biological systems. 
Indeed, since the 1960s, evolutionary computing has harnessed the principles of natural evolution as a general purpose search engine for solving challenging computational and engineering problems [cite - Back]. 
As evolutionary computing has matured, the field has identified common pitfalls in directing evolution and has developed a robust toolbox of algorithms to more effectively steer evolutionary processes.

As in evolutionary computing, directed evolution \textit{in vitro} begins with a library of variants (\textit{e.g.}, communities, genomes, or molecules). 
Variants are scored according to the phenotypic trait (or set of traits) of interest, and the variants with the `best' traits are selected and used to produce the next generation.
The method by which we select variants to propagate from generation to generation dramatically influences the success of directing evolution. 
In complex fitness landscapes, using only the overall `best' variants to produce the next generation can lead to premature convergence on local optima and adaptive stagnation; 
this is especially likely in scenarios with multiple objectives that have inherent functional trade-offs.
These pitfalls are well studied in evolutionary computation and have motivated new selection schemes that have dramatically improved the quality and diversity of evolved solutions.
% In conventional microbial directed evolution, only the most performant variants are selected to propagate [ cite - Sanchez].
% Does this prevent microbial directed evolution from realizing its full potential?

I plan to apply modern evolutionary computing selection algorithms (\textit{e.g.}, novelty-based algorithms, quality-diversity algorithms, lexicase-based algorithms, and multi-objective algorithms) to steer the evolution of microbial communities.
Eventually, we aim to further bridge digital and microbial experimental evolution by coevolving communities of digital organisms with microbial communities in real time, giving each influence over aspects of the other's environment.
For example, such a hybrid experimental evolution platform would provide the opportunity to investigate the \textit{de novo} evolution of feedbacks between different ecosystems.



% Modern evolutionary computing selection schemes promote diverse populations of candidate solutions that avoid adaptive stagnation and instead explore many independent evolutionary trajectories.
% At one extreme, \textit{novelty-based algorithms} disregard functional objectives entirely, selecting the most phenotypically novel variants [citations]; this drives the population to explore the fitness landscape and discover unique solutions.
% \textit{Diversity-maintenance algorithms} balance selecting for high-performance with selecting for diversity by adjusting the probability of selecting a variant based how many similar variants are already in the population [citations].
% \textit{Lexicase-based algorithms} evaluate variants on a set of tests 
% and choose parents based on their relative performance on random permutations of the full test set; this allows variants specializing on different sequences (or prioritizations) of tests to coexist until a complete solution evolves [citations]. 
% Finally, \textit{multi-objective evolutionary algorithms} (MOEAs) are designed for problems with multiple functional objectives that exhibit inherent trade-offs [citations]; MOEAs select and maintain Pareto-optimal variants, generating a repertoire of candidate solutions that prioritize different objectives.

% I plan to develop agent-based computational models of microbial community evolution featuring multiple levels of environmental complexity.
% These models will serve as a predictive testbed for parameterizing and applying evolutionary computing algorithms to produce target microbial community functions. 
% Despite overt similarities, evolutionary computing and biological directed evolution feature subtle differences that may impact overall evolutionary dynamics. 
% For example, in evolutionary computing, evaluation is independent of mutation, whereas microbial communities mutate, reproduce, and evolve continuously, even during evaluation. 
% Computational models allow me to test the efficacy of novelty-based, diversity-maintenance, lexicase-based, and multi-objective selection schemes under biologically realistic evolutionary conditions.

% \textbf{Implementing evolutionary algorithms \textit{in vitro}:}
% I will evolve microbial communities on a series of biodegradation tasks, comparing the efficacy of traditional artificial selection and evolutionary computing selection schemes (chosen using computational modeling experiments).
% Biodegradation is a well-established functional objective for microbial communities$^\citeSanchez$, 
% can easily be extended from single- to multi-objective with additional substrates, and
% provides inherent evolutionary trade-offs when degrading multiple compounds that use competing metabolic pathways.
% Recent advances in community-level metabolic modeling tools, such as COMETS$^\citeDukovski$, allow us to identify and approximate these metabolic trade-offs \textit{in silico}, providing further control over the magnitude of objective trade-offs. 
% I will determine the particular biodegradation tasks based on the accessibility of reagents and sample analysis technology (\textit{e.g.}, chromatography, spectroscopy, \textit{etc.}) in consultation with Zaman and the Metabolomics Core at the University of Michigan.
% I will use high-throughput Illumina sequencing with FREQ-Seq$^\citeChubiz$ to analyze how each selection regime affects genetic and community composition changes over time.
% My experiments will show whether computational evolutionary algorithms designed to maintain diversity and improve problem solving can effectively be transferred to the domain of directed microbial evolution.
% Most importantly, I will determine if these transferred selection schemes improve the prospects of \textit{in vitro} directed evolution.





% Eventually, we aim to bridge digital and microbial experimental evolution by coevolving communities of digital organisms with microbial communities in real time, giving each control over aspects of the other's environment.

% First, we aim to transfer ideas from evolutionary computing into the laboratory by implementing modern evolutionary computing algorithms in the context of directing the evolution of microbial populations. 
% Eventually, we aim to bridge digital and microbial experimental evolution by coevolving communities of digital organisms with microbial communities in real time, giving each control over aspects of the other's environment.
% For example, such a hybrid experimental evolution platform would provide the opportunity to investigate the \textit{de novo} evolution of feedbacks between different ecosystems.