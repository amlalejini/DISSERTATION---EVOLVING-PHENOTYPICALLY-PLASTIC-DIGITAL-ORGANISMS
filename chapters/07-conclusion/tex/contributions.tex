\section{Contributions}

In summary, this dissertation makes the following contributions:

% @AML: 'I' vs 'we' for this?

\begin{itemize}
    % --- Origins of plasticity ---
    \item In \textbf{Chapter \ref{chapter:evolutionary-origins-of-plasticity}}, I found that both environmental change rate and mutation rate influence the likelihood for adaptive phenotypic plasticity to evolve in populations of digital organisms. 
    By analyzing the lineages of plastic organisms, I identified unconditional trait expression and imperfect forms of phenotypic plasticity as important evolutionary building blocks for the evolution of adaptive plasticity. 
    
    % --- Consequences of plasticity ---
    \item In \textbf{Chapter \ref{chapter:evolutionary-consequences-of-plasticity}}, I used populations of digital organisms to empirically test whether the evolution of adaptive phenotypic plasticity alters evolutionary dynamics and influences evolutionary outcomes in cyclically changing environments.
    I found that the evolution of adaptive phenotypic plasticity stabilizes populations against environmental changes and constrains the rate of subsequent evolutionary change.
    By buffering populations against environmental change, adaptive plasticity improved novel trait retention and reduced the accumulation of deleterious mutations relative to non-plastic populations evolved in an otherwise identical environment. 
    
    % --- SignalGP ---
    \item In \textbf{Chapter \ref{chapter:signalgp}}, I introduced SignalGP, a novel genetic programming technique for evolving event-driven computer programs. 
    I showed that SignalGP allows us to evolve programs better able to rapidly interact the environment or with other agents. 
    % Further, SignalGP model digital organism. 
    
    % --- Tag-based genetic regulation ---
    \item In \textbf{Chapter \ref{chapter:tag-based-regulation}}, I developed tag-based genetic regulation, a new genetic programming technique that allows programs to dynamically adjust which code modules to express.
    I described how to augment existing genetic programming systems with tag-based regulation, and I showed that tag-based regulation improves problem-solving performance on context-dependent problems where programs must adjust how they respond to current inputs based on prior inputs.
    
    % --- Tag-accessed memory ---
    \item In \textbf{Chapter \ref{chapter:tag-accessed-memory}}, I proposed tag-accessed memory, a new mechanism for labeling and identifying memory positions in genetic programming.  
    With preliminary experiments, I found that, under favorable mutation rates, both tag-accessed memory and conventional direct-indexed memory achieve similar performance on a range of program synthesis problems. 
    % [Promising technique for future research].
    
\end{itemize}