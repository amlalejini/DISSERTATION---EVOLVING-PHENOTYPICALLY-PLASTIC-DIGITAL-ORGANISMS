

The capacity for adaptive phenotypic plasticity is an important characteristic of adaptive systems, including both biological organisms and solutions to computational problems (\textit{e.g.}, computer programs). 
In this dissertation, I used digital evolution experiments to explore the process by which adaptive plasticity evolves and to illuminate its effects on subsequent evolutionary dynamics.   
I have also demonstrated the value of phenotypic plasticity in the context of genetic programming, introducing novel techniques that allow us to evolve more dynamically responsive computer programs.