\section{Introduction}

% - Introduce genetic programming and automatic program synthesis + state what we do -
Genetic programming (GP) applies the natural principles of evolution to automatically synthesize programs rather than writing them by hand.
Indeed, the promise of automating computer programming has motivated advances in GP since its early successes in the 1980s \citep{forsyth_beagle_1981,cramer_representation_1985,koza_hierarchical_1989}. 
Just as human software developers have access to a dazzling array of programming languages, each specialized for solving different types of problems, GP features many ways to represent evolvable programs.
Each representation features different programmatic elements that vary in their syntax, organization, interpretation, and evolution.
These differences can dramatically influence the types of computer programs that can be evolved, and as such, influence a representation's problem-solving range \citep{hintze_buffet_2019,wilson_comparison_2008}. 
Here, we introduce and experimentally demonstrate tag-based module regulation for genetic programming, allowing us to more easily evolve programs capable of dynamically regulating responses to inputs over time.

%%%%%%%%%%%% lit to cite
% - TO READ (for different representations = different problem-solving):
%   - A Comparison of Cartesian Genetic Programming and Linear Genetic Programming
%%%%%%%%%%%%

% - Introduce value of software 'plasticity', link plasticity and modularity - 
Nearly all software applications are capable of conditionally responding to inputs.
For example, each input button on a calculator triggers a different software response; or, in the Small or Large problem from the Helmuth and Spector's automatic program synthesis benchmark suite \citep{helmuth_general_2015}, programs must output different classifications (``small'', ``large'', or ``neither'') depending on a numeric input value.
Just like such conditional logic is inherent in any non-trivial software, so to is it ubiquitous in biological organisms where it is referred to as ``plastic'' behavior or ``phenotypic plasticity.''
%Just like these dynamic responses are inherent in any non-trivial software, so to are they ubiquitous in natural organisms where they are referred to as ``plastic'' behaviors or ``phenotypic plasticity''.

% - link modular software design to improved software 'plasticity' -
Modular software design --- that is, designs that promote the partitioning and reusability of functional units --- is fundamental to good software development practices; this principle is all the more true in producing programs capable of complex ``plasticity.'' 
% Modular software design --- that is, designs that enable the separability and reusability of functional units --- is fundamental to good software development practices; this principle is all the more true in producing programs capable of complex ``plasticity''. 
By modularizing code (\textit{e.g.}, into functions, classes, libraries, \textit{etc.}), software developers can craft customized responses to inputs by composing relevant modules.  
These modules can each contain segments of code whose functionality would otherwise need to be reinvented for each response.
Likewise, modularity appears to be critical in natural genomes \citep{wagner_road_2007} as well as artificial evolving systems \citep{huizinga_does_2016}.
Moreover, evidence in these evolving systems suggests that modularity can improve the capacity for effective plasticity to arise \citep{londe_phenotypic_2015,ellefsen_neural_2015}. % and problem solving success?

Developing GP systems that facilitate the evolution of modular program architectures has long captured the attention of the genetic programming community. 
Koza introduced Automatically Defined Functions (ADFs) where callable functions can evolve as separate branches of GP syntax trees \citep{koza_genetic_1992,koza_genetic_1994}.
Angeline and Pollack developed compression and expansion genetic operators to automatically modularize existing code into libraries of parameterized subroutines \citep{angeline_evolutionary_1992}. 
Since these foundational advances, significant efforts have been made to allow GP representations to incorporate internal modules 
(\textit{e.g.},
\citep{spector_simultaneous_1996,oneill_grammar_2000,binard_abstraction-based_2007,walker_automatic_2008,spector_tag-based_2011,spector_tag-based_2012,lalejini_evolving_2018}),
to measure (and select for) modularity in evolving programs 
(\textit{e.g.}, \citep{krawiec_functional_2009,saini_modularity_2019,saini_using_modularity_2020}),
and to build ``libraries'' of reusable code modules accessible to evolving populations of programs
(\textit{e.g.}, \citep{rosca_learning_1994,banscherus_hierarchical_2001,keijzer_run_2004,keijzer_undirected_2005}).

% - Introduce more complicated form of plasticity -
These innovations have improved the ability of GP systems to link modules together to solve problems, thus improving their prospects as general-purpose tools for automatic program synthesis.
In existing GP work, links between modules, however, are typically hard coded and static during program execution; 
less is known for how to evolve programs that can adjust module associations on the fly.
For many types of problems, the appropriate set of modules to execute in response to a particular input changes over time.
This requires programs to continuously adjust associations between inputs and modular responses based on context.
For example, the computations that occur on a calculator after pressing the ``equals'' button are \textit{context-dependent}; that is, they depend on the set of operators and operands (\textit{i.e.}, inputs) previously provided. 
To achieve this design pattern, programs must internally track contextual information and typically regulate responses using explicit flow control directives (such as if-statements).
Our goal is to evolve programs that dynamically regulate modules during execution to more effectively solve context-dependent problems.  
To reach this goal, we draw inspiration from gene regulatory networks (both natural and artificial) to augment how program modules are called in GP.

% - Introduce what we did, summarize our findings -
Here, we propose to facilitate dynamic module composition by introducing tag-based module regulation for genetic programming. 
We extend existing tag-based naming schemes to allow programs to dynamically adjust associations between references and code modules. 
We experimentally demonstrate our implementation of tag-based genetic regulation in the context of SignalGP \citep{lalejini_evolving_2018}; however, our approach is immediately applicable to any existing tag-enabled GP system, such as tag-addressed Run Transferable Libraries \citep{keijzer_run_2004} or PushGP \citep{spector_tag-based_2011}.
We add ``regulation'' instructions to SignalGP that can adjust (\textit{i.e.}, promote or repress) which code modules respond to input signals and internal calls.
This extension allows evolution to structure a program as a gene regulatory network where genes are program modules and program instructions mediate regulation.
We show that module regulation improves problem-solving performance on problems where responses to particular inputs change depending on prior context (\textit{e.g.}, prior inputs). 
We also observe that our implementation of tag-based regulation can impede adaptive evolution when outputs are not context-dependent. 
% (\textit{i.e.}, the correct response to a particular input does not change over time).
