\section{Conclusion}

% - Positive results, immediately applicable to other tag-enabled systems -
We demonstrated that tag-based genetic regulation allows GP systems to evolve programs with more dynamic plasticity.
These evolved programs are better able to solve context-dependent problems where the appropriate software modules to execute in response to a particular input changes over time.
Genetic regulation broadens the applicability of SignalGP, both as a representation for problem-solving and as a type of digital organism for studying evolutionary dynamics \citep{Lalejini_Moreno_Ofria_DISHTINY_2020}.
Further, this work illustrates an approach for easily incorporating tag-based models of gene regulation into existing GP systems. 

% - negative results, future work on matching schemes -
Our results also reveal that tag-based regulation is not necessarily beneficial across all problem domains.
We observed that the addition of tag-based regulation can impede adaptive evolution on problems where responses to inputs are not context-dependent (\textit{e.g.}, the independent-signal task and postfix version of the Boolean-logic calculator problem). 
A more thorough examination of what types of context-free problems are most sensitive to tag-based regulation---and how to mitigate any harm---would be potentially fruitful.

% ---- BEGIN REVISION ----
% - issues with existing representations -
Across all problems used in this work, the tag representation and matching scheme that we used was clearly sufficient for success. 
However, existing tag systems are limited in their capacity to scale up to substantially larger gene regulatory networks. 
As these networks grow, the specificity required for references to differentiate between modules increases. 
At some point references become brittle, as any mutation will switch the module that a call triggers.
In ongoing work, we are investigating the wide variations in scalability of different metrics for measuring the similarity between tags.
Substantial work will also need to be conducted by the community in order to develop more scalable representations for tag-based naming.
% @AML: new line below (currently very weak sentence)
For example, insights from the indirect referencing mechanisms of artificial biochemical networks and enzyme genetic programming systems may prove to be informative in developing new tag representations \citep{lones_modelling_2004,lones_artificial_2014,lones_biochemical_2013}.
% --- END REVISION ----

% ---- BEGIN REVISION ----
% - Comment: Another thing that would be good to comment on is the interpretability of the evolved solutions. I'm guessing that the regulatory-based solutions are harder for humans to understand?
% @AML: needs work!!
Evolved programs are often more challenging to read and understand than programs written by human developers.
In our experience, evolved programs that make use of tag-based regulation were substantially more difficult to read and interpret by hand than evolved programs that do not use tag-based regulation.
We found that visualizations of tag-based regulatory networks and program execution traces (\textit{e.g.}, Figures \ref{chapter:tag-based-regulation:fig:signal-counting-example-networks} and \ref{chapter:tag-based-regulation:fig:boolean-calc-prefix-example-networks}) improved our ability to understand how a given evolved program worked.
As we scale up tag-based regulation, the development of interactive visualizations will become increasingly important for understanding evolved programs that make use of tag-based regulation.  
% ---- END REVISION -----

The current investigations have focused on regulation as a problem-solving tool, but with a few extensions these sorts of systems can also help us answer open questions about biological evolution.
Our current implementation of tag-based regulation facilitates plasticity only within a program's lifetime; if we extend this capacity across multiple generations, we can study the effects of epigenetic inheritance on evolutionary dynamics.
Epigenetic inheritance refers to heritable phenotypic changes that are not directly encoded by the underlying genetic sequence \citep{bender_plant_2002,jablonka_transgenerational_2009}.
For example, epigenetics is used in combination with gene regulation for cell-type differentiation in multicellular organisms \citep{mohn_genetics_2009, smith_dna_2013} and caste determination in some species of eusocial insects \citep{weiner_epigenetics_2012}.
SignalGP supports epigenetics with the addition of instructions that mark existing function regulation as heritable.
For our next steps, we will apply epigenetics-enabled SignalGP to study fraternal transitions in individuality and the evolution of differentiation before, during, and after a transition occurs \citep{Lalejini_Moreno_Ofria_DISHTINY_2020}.
Open-ended experiments with epigenetics and gene regulation will help illuminate the relationship between within-lifetime plastic adaptation and evolutionary adaptation over generational time scales.
Additionally, mechanisms for epigenetic inheritance have been shown to potentially improve GP performance \citep{la_cava_genetic_2015, la_cava_inheritable_2015,ricalde_evolving_2017}; as such, we plan to apply our insights back to automatic program synthesis. 