
\begin{table}[h!]
    \small
    \centering
    \begin{tabular}{c | c | c}
        \toprule
        Test case ID & Input Sequence & Correct Response \\ \hline 
        0 &
        S-0, S-0 &
        Response-A \\ 
        % \hline
        1 &
        S-0, S-1 &
        Response-B \\  
        % \hline
        2 &
        S-0, S-2 &
        Response-C \\  
        % \hline
        3 &
        S-0, S-3 &
        Response-D \\ 
        % \hline
        
        4 &
        S-1, S-0 &
        Response-B \\ 
        % \hline
        5 &
        S-1, S-1 &
        Response-C \\ 
        % \hline
        6 &
        S-1, S-2 &
        Response-D \\ 
        % \hline
        7 &
        S-1, S-3 &
        Response-A \\ 
        % \hline
        
        8 &
        S-2, S-0 &
        Response-C \\ 
        % \hline
        9 &
        S-2, S-1 &
        Response-D \\ 
        % \hline
        10 &
        S-2, S-2 &
        Response-A \\ 
        % \hline
        11 &
        S-2, S-3 &
        Response-B \\ 
        % \hline
        
        12 &
        S-3, S-0 &
        Response-D \\ 
        % \hline
        13 &
        S-3, S-1 &
        Response-A \\ 
        % \hline
        14 &
        S-3, S-2 &
        Response-B \\ 
        % \hline
        15 &
        S-3, S-3 &
        Response-C \\ 
        \bottomrule
    \end{tabular}
    \caption{\small 
    \textbf{Input signal sequences for the contextual-signal problem.} 
    }
    \label{chapter:tag-based-regulation:tab:context-signal-input-combos}
\end{table}