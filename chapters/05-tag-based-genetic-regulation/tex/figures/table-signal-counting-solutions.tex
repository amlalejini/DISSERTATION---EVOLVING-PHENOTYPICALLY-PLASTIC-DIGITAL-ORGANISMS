
\begin{table}[ht!]
    \small
    \centering
    \begin{tabularx}{\columnwidth}{cYY}
        \toprule
          & Regulation-off condition 
          & Regulation-on condition \\
        %  \hline
        \cmidrule(lr){1-3}
         Two-signal   
            & \RSPTwoSigMemOnlySolutionCnt  % Memory-only
            & \RSPTwoSigBothSolutionCnt \\  % Both
        %  \hline
         Four-signal
            & \RSPFourSigMemOnlySolutionCnt  % Memory-only
            & \RSPFourSigBothSolutionCnt \\  % Both
        %  \hline
         Eight-signal 
            & \RSPEightSigMemOnlySolutionCnt  % Memory-only
            & \RSPEightSigBothSolutionCnt \\  % Both
        %  \hline
         Sixteen-signal
            & \RSPSixteenSigMemOnlySolutionCnt  % Memory-only
            & \RSPSixteenSigBothSolutionCnt \\  % Both
         \bottomrule
    \end{tabularx}
    \caption{\small
    \textbf{Signal-counting problem-solving success.}
    This table gives the number of successful replicates (\textit{i.e.}, in which a perfect solution evolved) out of 200 on the signal-counting problem across four problem difficulties and two experimental conditions. 
    For each problem difficulty, the regulation-off condition was less successful than the regulation-on condition (Fisher's exact test; all difficulties: $p < 10^{-15}$).
    }
    \label{chapter:tag-based-regulation:tab:signal-counting-solutions}
\end{table}


% two-signal: p-value < 2.2e-16, 
% four-signal: p-value < 2.2e-16
% eight-signal: p-value < 2.2e-16
% sixteen-signal: p-value < 2.2e-16