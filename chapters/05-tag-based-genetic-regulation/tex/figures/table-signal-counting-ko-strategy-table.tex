\begin{table}[htbp]
    \small
    \centering
    \begin{tabularx}{\columnwidth}{cYYY}
        \toprule
          & No regulation required
          & Regulation required 
          & Unsolved \\
        %  \hline
        \cmidrule(lr){1-4}
         Two-signal     
            & \RSPTwoSigMemOnlyStrategyCnt  % Memory-only
            & \RSPTwoSigRegRequiredStrategyCnt  
            & \RSPTwoSigUnsolvedCnt \\
        %  \hline
         Four-signal    
            & \RSPFourSigMemOnlyStrategyCnt  % Memory-only
            & \RSPFourSigRegRequiredStrategyCnt   % 
            & \RSPFourSigUnsolvedCnt \ \\
        %  \hline
         Eight-signal   
            & \RSPEightSigMemOnlyStrategyCnt  % Memory-only
            & \RSPEightSigRegRequiredStrategyCnt  % 
            & \RSPEightSigUnsolvedCnt \\
        %  \hline
         Sixteen-signal 
            & \RSPSixteenSigMemOnlyStrategyCnt  % Memory-only
            & \RSPSixteenSigRegRequiredStrategyCnt  % 
            & \RSPSixteenSigUnsolvedCnt \\
         \bottomrule
    \end{tabularx}
    \caption{\small 
    \textbf{Mechanisms underlying solutions from the regulation-on condition for the signal-counting problem.}
    To determine a successful program's underlying strategy, we re-evaluated the program with global memory access instructions knocked out (\textit{i.e.}, replaced with no-operation instructions) and with regulation instructions knocked out.
    This table shows the number of regulation-on solutions that actually rely on regulation to solve the signal-counting problem. 
    }
    \label{chapter:tag-based-regulation:tab:signal-counting-ko-strategies}
\end{table}