\section{Conclusion}

Our preliminary experiments show that, under favorable mutation rates, both tag-accessed and direct-indexed memory achieve statistically equivalent performance.
Because tag-based instruction arguments index into the \textit{closest matching} memory register, single bit-flip mutations may be neutral (not affecting the program's behavior), which affords programs robustness to minor genetic perturbations. 
The down-side to a more robust genetic encoding for instruction arguments is that mutations are less able to generate novel phenotypic variation (program behavior).
For the relatively simple program synthesis problems used in our experiments, the capacity of our GP system to generate novel phenotypic variation is likely more important than robustness to mutation. 
Future work will continue to explore the efficacy of tag-accessed memory, supplementing bit-flip mutation operators with more impactful mutation operators that allow tag-mutations to more easily generate novel phenotypic variation.
Future work will also investigate the possibility of coevolving register labels (tags) with programs, allowing evolution to adjust the adjacency of memory registers in tag-space.