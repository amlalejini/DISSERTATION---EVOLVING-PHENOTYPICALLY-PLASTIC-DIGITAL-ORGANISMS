\section{Conclusion}

In this work, we evolved populations of phenotypically plastic organisms at varied rates of environmental fluctuation and mutation using the Avida Digital Evolution Platform. 
We analyzed the lineages of evolved genotypes for clues about the evolutionary stepping stones toward phenotypic plasticity. 
We found that the capacity for phenotypic plasticity evolved under conditions identified by previous research \citep{clune_investigating_2007,ghalambor_behavior_2010}. 
We found evidence that traits are generally expressed unconditionally prior to the evolution of conditional trait expression and that sub-optimal forms of phenotypic plasticity generally evolve before optimal forms of phenotypic plasticity. 
Both of these results are examples of evolution's use of simpler functions as building blocks for more complex functions as in Lenski \textit{et al.} \citep{lenski_evolutionary_2003}. 

Visual inspection of the evolutionary histories leading to phenotypically plastic organisms suggests that under certain conditions stochastic phenotype switching evolves as an alternative strategy to phenotypic plasticity, just as it does in many bacteria \citep{moxon_bacterial_2006,rainey_evolutionary_2011}.  
Of course, in these bacterial cases, hypermutable sites tend to appear in the genomes (called ``contingency loci'') that facilitate such task switching.

Given these promising results, we plan to explore whether stochastic phenotype switching can be a viable evolutionary strategy in the absence of the ability to evolve hypermutable regions of the genome. 
Given the potential difficulty in maintaining the necessary genetic machinery associated with phenotypic plasticity, are there cases in which stochastic phenotype switching is more robust than phenotypic plasticity? 
And, does this contribute to the evolution of stochastic phenotype switching as an evolutionary strategy? 
Metrics are clearly needed for quantifying stochastic phenotype switching in digital systems and for evaluating the mutational landscapes of genotypes along a lineage. 