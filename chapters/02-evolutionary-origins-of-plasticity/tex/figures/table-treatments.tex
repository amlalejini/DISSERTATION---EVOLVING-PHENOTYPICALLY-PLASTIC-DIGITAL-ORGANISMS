% Table 'bout treatments
\begin{table}
\renewcommand{\arraystretch}{1.5}
  \centering
  \small
  \begin{tabular}{| >{\raggedright} p{0.3\columnwidth} |r | r| }
    \hline
    \centering \textbf{Treatment} & \multicolumn{1}{>{\raggedright\arraybackslash}p{0.25\columnwidth}|}{\centering \textbf{Point-mutation Rate}} & \multicolumn{1}{>{\raggedright\arraybackslash}p{0.30\columnwidth}|}{\centering \textbf{Environment Cycle Length }}
    \\ \hline 
    Baseline &  0.0075 &  100 updates  
    \\ \hline 
    Low Mutation Rate &  0.0025 & 100 updates 
    \\ \hline 
    High Mutation Rate &  0.0125 &  100 updates 
    \\ \hline
    Short Environment Cycle Length & 0.0075 &  50 updates 
    \\ \hline 
    Long Environment Cycle Length & 0.0075 &  200 updates 
    \\ \hline 
  \end{tabular}
  \caption{\small 
  \textbf{Differences among the five experimental treatments. }
  Point-mutation rate is given as mutations per instruction copied.
  Environment cycle length describes the length of time (in updates) an environment is active before toggling to the alternative environment.
  }
  \label{chapter:origins-of-plasticity:table:treatments}
\end{table}