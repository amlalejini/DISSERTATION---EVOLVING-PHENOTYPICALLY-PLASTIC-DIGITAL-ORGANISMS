\section{Introduction}
\label{chapter:origins-of-plasticity:sec:introduction}

% What is phenotypic plasticity and why is it important?
Phenotypic plasticity is the capacity for a genotype to express different phenotypes in response to different environmental conditions \citep{ghalambor_behavior_2010} and is ubiquitous throughout nature. 
Phenotypic plasticity is central to many complex traits and developmental patterns found in nature and often serves as a key strategy employed by organisms to respond to spatially and temporally variable environments \citep{bradshaw_evolutionary_1965,murren_constraints_2015}.
For example, \textit{Daphnia pulex} use plasticity to differentially invest in morphological defenses during development, depending on the presence of predators in their local environment \citep{black_demographic_1990}. 
Genetically homogeneous cells in a developing multicellular organism leverage their capacity for phenotypic plasticity to coordinate their expression patterns through environmental signals \citep{schlichting_origins_2003}.
Thus, understanding the evolution of plasticity is an important step toward a deeper understanding of biological complexity. 

Phenotypic plasticity also has practical applications in the field of evolutionary computation where evolution by natural selection is harnessed to solve challenging computational and engineering problems. 
In many realistic problem domains, conditions are noisy or cyclically change. 
Plasticity enables solutions to dynamically respond to changing problem conditions and be robust to noise. 
Both the biological and evolutionary computation domains motivate the following questions: 
(1) Under what conditions does phenotypic plasticity evolve? 
And (2), what are the evolutionary stepping stones for phenotypic plasticity? 

% What are the conditions for phenotypic plasticity?
Ghalambor \textit{et al.} identify four conditions that are necessary for phenotypic plasticity to evolve: 
(1) populations are exposed to temporally or spatially varying environments, 
(2) the environments are differentiable by reliable signals, 
(3) different environments favor different phenotypes, and 
(4) no single phenotype can exhibit high fitness across all environments \citep{ghalambor_behavior_2010}. 
Theoretical and empirical findings support that phenotypic plasticity can evolve under these conditions in both natural and artificial systems \citep{clune_investigating_2007,goldsby_evolution_2010,goldsby_evolutionary_2014,hallsson_selection_2012,nolfi_phenotypic_1994}.

% Process by which phenotypic plasticity evolves, motivate digital evolution
In addition to exploring the conditions that facilitate the evolutionary origin of phenotypic plasticity, it is also important to explore the step-by-step process out which plasticity actually evolves. 
What are the reoccurring themes as evolution progresses toward more plastic strategies?
Are there genotypic or phenotypic patterns present in lineages leading to phenotypically plastic organisms? 
These types of questions are especially difficult to address in laboratory systems due to the slow pace of natural evolution, imperfections in lineage tracking, and the difficulty of acquiring high-resolution data on genotypes and phenotypes. 
As such, artificial life systems are the most effective way to observe and analyze the process by which phenotypic plasticity evolves.

% What did we do? 
Here, we use the Avida Digital Evolution Platform \citep{ofria_avida:_2009} to explore the process by which phenotypic plasticity evolves in a fluctuating environment. 
First, we investigate how environmental factors impact the evolution of phenotypic plasticity.
Specifically, we evaluate how mutation rate and environment fluctuation rate affect the evolution of adaptive plasticity.
Next, we identify the evolutionary stepping stones in the evolution of adaptive phenotypic plasticity: do digital organisms evolve to express traits unconditionally before evolving to conditionally express them as a function of their environment, and do sub-optimal forms of plasticity evolve before more optimal forms of plasticity? 
We empirically test whether such intermediate evolutionary stepping stones are important to the evolution of adaptive plasticity. 
Finally, we examine alternative evolutionary strategies to phenotypic plasticity in fluctuating environments and see evidence for bet-hedging strategies that use mutationally induced phenotype switching as a substitute for sensory-dependent plasticity.   