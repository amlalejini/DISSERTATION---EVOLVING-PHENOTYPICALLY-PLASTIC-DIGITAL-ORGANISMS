\section{Thesis Statement}

% Phenotypic plasticity is critical for biological organisms and computer programs that need to dynamically respond to complex and ever-changing environments.
% Thus, 
% Phenotypic plasticity is a critical characteristic of adaptive systems, which must dynamically respond to complex and ever-changing environments.
% Digital evolution studies help us to understand how plasticity evolves, its subsequent evolutionary consequences, and how to harness it for automatic program synthesis.

Adaptive systems require phenotypic plasticity to dynamically respond to complex and ever-changing environments.
We must study digital evolution and genetic programming systems if we are to understand how plasticity evolves, how it shapes subsequent evolutionary outcomes, and how to harness it to synthesize adaptive computer programs. %for automatic program synthesis.


% Plasticity useful.

% Plasticity is critical for organisms needing to respond to complex and ever changing environments as well as for computer programs that need to respond to complex and ever changing environments.
% Exploring the ramifications of plasticity: how does it effect subsequent dynamics and if we can have modularized responsiveness that allow useful applied goals/allow us to solve problems that would otherwise be challenging?

% Perspective I'm taking in this dissertation
% plasticity critical component for dynamic adaptiv systems, need to understand how it evolve, effects and how to best harness it applied systems.