\section{Contributions}

% [Discuss chronological ordering => development as an academic/scientist].

This dissertation can be divided into two parts.
In part one (Chapters \ref{chapter:evolutionary-origins-of-plasticity} and \ref{chapter:evolutionary-consequences-of-plasticity}), I conducted digital evolution studies to investigate the evolutionary origins and consequences of adaptive phenotypic plasticity in cyclic environments.
In the part two (Chapters \ref{chapter:signalgp}, \ref{chapter:tag-based-regulation}, and \ref{chapter:tag-accessed-memory}), I introduce and experimentally demonstrate three novel genetic programming techniques for representing and evolving more responsive and adaptive (\textit{i.e.}, plastic) computer programs: signal-driven genetic programs (SignalGP), tag-based genetic regulation, and tag-accessed memory.

\subsection{Part 1. Understanding the evolutionary origins and consequences of adaptive phenotypic plasticity in fluctuating environments}

% -- Origins --
% ABSTRACT 
\textbf{Chapter \ref{chapter:evolutionary-origins-of-plasticity}} focuses on the step-by-step process by which adaptive phenotypic plasticity evolves in a fluctuating environment.
Many effective and innovative survival mechanisms used by natural organisms rely on the capacity for phenotypic plasticity.
Understanding the evolution of phenotypic plasticity is an important step toward understanding the origins of many types of biological complexity, as well as to meeting challenges in evolutionary computation where dynamic solutions are required.
In Chapter \ref{chapter:evolutionary-origins-of-plasticity}, I used the Avida Digital Evolution Platform to experimentally explore the selective pressures and evolutionary pathways that lead to phenotypic plasticity.  
I present evolved lineages wherein unconditionally expressed (non-plastic) traits tend to evolve first.
Next, imprecise forms of phenotypic plasticity often appear before optimal forms finally evolve.   
I experimentally disallowed each of these intermediate phenotypes to test their importance.
I found that phenotypic plasticity is most likely to evolve when both unconditional trait expression and sub-optimal forms of plasticity are allowed to evolve first.
I also show that both mutation rate and environmental change rate influence the evolution and maintenance of adaptive phenotypic plasticity.

% We found that under all conditions, populations can fail to evolve phenotypic plasticity, instead relying on mutation-based solutions.

% @AML: todo - update this depending on 
In \textbf{Chapter \ref{chapter:evolutionary-consequences-of-plasticity}}, I used Avida to investigate how the evolution of adaptive phenotypic plasticity alters evolutionary dynamics and influences evolutionary outcomes in cyclically changing environments.
% --- can cut next sentence ---
Specifically, I 
(1) examined the evolutionary histories of plastic and non-plastic populations to test whether the evolution of adaptive plasticity promotes or constrains subsequent evolutionary change;
(2) evaluated how adaptive plasticity influences fitness landscape exploration and exploitation by testing whether plastic populations are better able to evolve and then maintain novel traits;
and (3) tested if the evolution of adaptive plasticity increases the potential for deleterious mutations to accumulate in evolving genomes.
% ------
I found that populations with adaptive phenotypic plasticity evolve more slowly than non-plastic populations, which rely on genetic variation from \textit{de novo} mutations to continuously re-adapt to the environment.
The non-plastic populations undergo more frequent selective sweeps and accumulate many more genetic changes. %, [finish listing big results]. 
I find that phenotypic plasticity stabilizes populations against environmental fluctuations; whereas the repeated selective sweeps in non-plastic populations drive the loss of beneficial traits and accumulation of deleterious mutations via genetic hitchhiking.  
As such, plastic populations are more likely to retain novel adaptive traits than their non-plastic counterparts. 
My findings suggest that the stabilizing effect of adaptive phenotypic plasticity plays an important role in subsequent adaptive evolution.
Indeed, evolutionary dynamics in adaptively plastic populations was more similar to that of populations evolving in a static environment than to that of non-plastic populations evolving in an identical fluctuating environment. 


% Indeed, the evolution of phenotypic plasticity shifted many dynamics to be more similar to that of populations evolving in a static environment than that of non-plastic populations evolving.

% @AML: need better heading here
\subsection{Part 2. Building more responsive program representations}

% -- Lead-in --
% Motivate
% MODEL DIGITAL ORGANISMS?? Pull from abstract for ECLife?
In traditional digital evolution systems (\textit{e.g.}, Tierra and Avida), genetic programs---linear sequences of program instructions---are expressed procedurally: actions are performed sequentially, and programs must explicitly check for new sensory information before they can react.
These linear program representations are convenient for their simplicity to analyze, but do not easily support the evolution of modularized responses to environmental signals that can be dynamically regulated over the organism's lifetime.
This shortcoming holds conventional digital organisms back as model systems for studying the evolution of complex forms of phenotypic plasticity.
Likewise, it also limits conventional linear genetic programming systems from evolving modular programs capable of dynamically regulating responses to inputs over time. 

% ------------------------------------------------------------------------------
% --- @AML: think about incorporating the following text:
% However, despite digital evolution's track record, current forms of digital organisms lack a rich spectrum of mechanisms for interacting with other organisms or for responding to environmental changes, which limits their capacity to study the evolution of biological responsiveness.
% Traditional digital organisms generally follow an imperative programming paradigm where computation is driven procedurally. 
% Program execution starts at the top of the program and proceeds in sequence, instruction-by-instruction, jumping or branching as dictated by executed instructions \citep{mcdermott_genetic_2015}.
% Thus, unlike natural organisms, exogenous signals cannot directly trigger computations (\textit{e.g.}, as it does in cellular signal transduction) in traditional forms of digital organisms.
% Instead, these digital organisms must actively (via repeated polling) monitor environmental cues or communication from other agents.
% ------------------------------------------------------------------------------

In Chapters \ref{chapter:signalgp}, \ref{chapter:tag-based-regulation}, and \ref{chapter:tag-accessed-memory}, we introduce novel genetic programming techniques that both improve the problem-solving potential of genetic programming systems and provide new forms of model digital organisms for \textit{in silico} experimental evolution.
This work helps both digital evolution and genetic programming systems realize a richer spectrum of evolutionary outcomes that more closely rivals that of biological evolution.

% -- SignalGP --
In \textbf{Chapter \ref{chapter:signalgp}}, I present SignalGP, a new genetic programming technique designed to incorporate the event-driven programming paradigm into computational evolution's toolbox. 
Event-driven programming is a software design philosophy that simplifies the development of reactive programs by automatically triggering program modules (event-handlers) in response to external events, such as signals from the environment or messages from other programs. 
I demonstrate the value of the event-driven paradigm using two distinct test problems (an environment coordination problem and a distributed leader election problem) by comparing SignalGP to variants that are otherwise identical, but must actively query sensors to process events or messages. 
In each of these problems, responsiveness to the environment or other agents is critical for maximizing fitness. 
I also discuss ways in which SignalGP can be generalized beyond a linear GP context.

% -- Tag-based regulation --
In \textbf{Chapter \ref{chapter:tag-based-regulation}}, I introduce and experimentally demonstrate tag-based genetic regulation, a new genetic programming technique that allows programs to dynamically adjust which code modules to express.
This extension allows evolution to structure a program as a gene regulatory network where modules are can be made more or less accessible based on instruction executions.
I find that tag-based regulation improves problem-solving performance on context-dependent problems; that is, problems where programs must adjust how they respond to current inputs based on prior inputs (\textit{i.e.}, current context).
Indeed, some context-dependent problems were unable to be solved by the system until regulation was added.
I also identify scenarios where the correct response to a particular input never changes, rendering tag-based regulation an unneeded functionality that can impede adaptive evolution.
Tag-based genetic regulation broadens our repertoire of techniques for evolving more dynamic genetic programs and can easily be incorporated into existing tag-enabled GP systems.

% -- Tag-accessed memory --
Finally, in \textbf{Chapter \ref{chapter:tag-accessed-memory}}, I briefly demonstrate the use of tags to label memory positions in GP, enabling programs to define and use evolvable variable names \citep{lalejini_tag-accessed_2019}.
My tag-based memory implementation did not substantively affect problem-solving performance across several program synthesis benchmark problems.
However, tag-based addressing features a larger addressable memory space than more traditional register-based memory approaches in GP.
Further, in combination with tag-based regulation, tag-accessed memory has the potential to enable more dynamic, context-dependent memory storage in GP.