\section{Genetic programming}
% -- Very high-level literature review --

% -- What is genetic programming? --

% --- Relationship to GP ----
% @AML: needs better transition in/out of this paragraph
Both digital evolution (as used in this dissertation) and genetic programming systems evolve populations of computer programs.
Conventional genetic programming systems evolve computer programs by [following] the five steps: 
(1) random candidate solutions,
(2) evaluate each candidate's quality relative to one or more criteria,
(3) select promising candidates based their quality,
(4) vary candidates by mutating or recombining them, and
(5) repeat this process from step 2 until a sufficiently good solution is found [cite - Ofria 2009].
This approach works well for solving problems, but omits many important aspects of living systems.
For example,
[self-replication, whereas gp systems perform this automatically (some work has been done in gp [cite - autoconstructive evolution])].
[interact with other organisms and their environment, which is often not part of algorithm, or these interactions are very limited].
As such, genetic programming systems are not an ideal platform for conducting experimental evolution \textit{in silico}. 


% -- forms/representations (non-exhaustive) --
%   @AML: how to make it clear that this is non-exhaustive
% - tree-based gp (pioneered by Koza?, earliest and perhaps must widespread and successful)
% - graph-based gp (CGP, TPG)
% - grammatical evolution
% - stack-based gp
% - linear gp

% -- application areas --
% - symbolic regression
% - bug fixing
% - program synthesis

% -- benefits --
% - transparency
% - efficient => computers are designed to run computer programs, so GP can run on existing hardware at a very low level
% - ??

% -- my work --
% - In my work, I focus on making gp systems more dynamic.
% - (1) modules with evolvable labels
% - (2) allowing signals to automatically trigger modules
% - (3) allowing programs to adjust links between modules on the fly
% - (4) developing more flexible ways of memory access





% ---------------------------------------------------------
% Independent origins and different approaches
% - Tied together by the fact that we're using Darwinian dynamics to generate programs.

% Digital evolution has its roots in Genetic  Programming(GP), wherein computer programs are evolved using naturalprinciples. 
% For example, Avida (Ofria et al., 2009) is a pop-ular digital evolution system that uses self-replicating lineargenetic programs as its organisms. The organisms generallyfollow an imperative programming paradigm where computation is driven procedurally. 
% Program execution starts at thetop of the program and proceeds in sequence,  instruction-by-instruction, jumping or branching as dictated by executedinstructions (McDermott and O’Reilly, 2015)