% -- division of labor --
\citep{goldsby_evolution_2010}
evolved groups of clonal organisms capable of differentiation and  division of labor
Observed two mechanisms, whose frequency of use depending on task: location awareness and intra-group communication
task switching costs promoted the evolution of division of lbor
cooperative problem decomponsition where group members not only divided labor amongst themselves but collaborated to complete individual tasks


\citep{goldsby_task-switching_2012}
Our results demon- strate that as task-switching costs rise, groups increasingly evolve division of labor strategies. We analyze the mechanisms by which organisms coordinate their roles and discover strategies with striking biological parallels, including communication, spatial patterning, and task-partitioning behaviors. In many cases, under high task-switching costs, individuals cease to be able to perform tasks in isolation, instead requiring the context of other group member


Other group coordination tasks 
% - synchronization
\citep{knoester_evolution_2011}
plasticity based on sensory information and age
volution of temporal behavior, specifically synchronization and desynchronization, through digital evolution and group selection.
The evolved solutions utilized an adaptive frequency strategy, similar to models of firefly behavior, that altered their control flow based on a combination of sensed information and the organism's age. 
provided with a similar mechanism and minimal information about their environment, are capable of evolving algorithms for synchronization and desynchronization, and that the evolved behaviors are robust
to message loss
% - consensus
\citep{knoester_genetic_2013}
In this paper, we describe a study of the evolution of consensus, a cooperative behavior in which members in both homogeneous and heterogeneous groups, must agree on information sensed in their environment.
homogeneous groups best at consensus
local communication

% suppressed rate of self-replication based on local density
\citep{beckmann_evolving_2009}
we demonstrate the evolution of a quorum sensing behavior in populations of digital organisms. Specifically, we show that digital organisms are capable of evolving a strategy to collec- tively suppress self-replication, when the population density reaches a specific, evolved threshold.
Specifically, we observed Avida populations that evolved a group communication behavior to inhibit self-replication once a deme has reached a density threshold.
\citep{beckmann_evolution_2012}
Quorum sensing (QS) is a collective behavior whereby actions of individuals depend on the density of the surrounding population.
 we evolve QS in populations of digital organism
populations are capable of evolving a strategy to collectively suppress self-replication when the population density reaches an evolved threshold
observed resistance to cheating/quorum quelching strategies


\citep{johnson_more_2014}
We found that quorum- sensing altruists killed a greater number of competitors per explosion, winning competitions against non-communicative altruists. These findings indicate that quorum sensing could increase the beneficial effect of altruism and the suite of con- ditions under which it will evolve.
When an organism explodes, it kills competitors, and thus behaves similarly to colicinogenic E. coli. We study whether such organisms will evolve to use quorum-sensing capabilities and whether these capabilities provide a compet- itive advantage against altruists who cannot communicate.


\citep{wagner_behavioral_2014}
 Prey adaptive use of sensors for finding food im- proves evolved predator-avoidance skills.
 ndeed, prey coevolved with predators demonstrated a substantial re- liance on information about their environment in making foraging decisions, and increasingly so over evolutionary time (Fig. 3):
prey evolved with predators proved to be more adept and competitive in foraging than prey evolved without predators, 

% -- navigation (spatial variability) --
\citep{elsberry_cockroaches_2009}
% - not a great paper, not totally clear that organisms used sensors (I think they had to)
% follow/exploit a resource gradient 
evolution of movement strategies in a model environment with a single local resource that diffuses to produce a gradient, which organisms have the ability to follow.
 to retrieve information from the environment about local differences in resource availability
 
 
% -- associative learning --
\citep{pontes_investigations_2017}
% associate cues with responses  within lifetime to navigate a nutrient path
evolved associative learning in navigation task
We evolved organisms capable of associating new cues with their meaning in different contexts.

\citep{pontes_evolutionary_2019}
Starting with a non-learning, sessile ancestor
Environmental patterns that are stable across generations foster the evolution of reflexive behavior, while environmental patterns that vary across generations, but remain consistent for periods within an organism’s lifetime, foster the evolution of learning behavior
we observe that an intrinsic value system evolves alongside behavior and supports associative learning by providing reinforcement for behavior conditioning.
