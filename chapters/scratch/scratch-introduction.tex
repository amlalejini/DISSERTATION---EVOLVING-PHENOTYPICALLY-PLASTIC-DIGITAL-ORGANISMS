
In general

Many traditional linear genetic programming representations.

My work with self-replicating computer programs in Avida allowed me to build intuition for how mechanisms of phenotypic plasticity are implemented 

new model digital organisms capable of more complex forms of phenotypic plasticity
and new techniques for representing computer programs in genetic programming

My work studying phenotypic plasticity revealed shortcomings in the way that phenotypic plasticity is typically encoded in linear programs 

linear genetic programs 

Work with plasticity Avida revealed shortcomings of way that traditional digital organisms (and traditional linear programs in gp) are represented. 
In Avida, plasticity...
Challenging for complex regulation we observe in nature [citations].


This approach works well for solving problems, but omits many important aspects of liv ing systems.
For example,
[self-replication, whereas gp systems perform this automatically (some work has been done in gp [cite - autoconstructive evolution])].
[interact with other organisms and their environment, which is often not part of algorithm, or these interactions are very limited].
As such, genetic programming systems are not an ideal platform for conducting experimental evolution \textit{in silico}. 

% ---------------------------------------------------------
% Independent origins and different approaches
% - Tied together by the fact that we're using Darwinian dynamics to generate programs.

% Digital evolution has its roots in Genetic  Programming(GP), wherein computer programs are evolved using naturalprinciples. 
% For example, Avida (Ofria et al., 2009) is a pop-ular digital evolution system that uses self-replicating lineargenetic programs as its organisms. The organisms generallyfollow an imperative programming paradigm where computation is driven procedurally. 
