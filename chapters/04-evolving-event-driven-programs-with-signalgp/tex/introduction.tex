\section{Introduction}

% SignalGP exists; we did it!
Here, we introduce SignalGP, a new genetic programming (GP) technique designed to provide evolution direct access to the event-driven programming paradigm, allowing evolved programs to handle signals from the environment or from other agents in a more biologically inspired way than traditional GP approaches. 
In SignalGP, signals (\textit{e.g.}, from the environment or from other agents) direct computation by triggering the execution of program modules (\textit{i.e.}, functions). 
SignalGP augments the tag-based referencing techniques demonstrated by Spector \textit{et al.}
\citep{spector_tag-based_2011,spector_whats_2011,spector_tag-based_2012} to specify which function is triggered by a signal, allowing the relationships between signals and functions to evolve over time. 
The SignalGP implementation presented here is demonstrated in the context of linear GP, wherein programs are represented as linear sequences of instructions; however, the ideas underpinning SignalGP are generalizable across a variety of genetic programming representations. 

% Linear GP is imperative. Event-driven computing is different.
Linear genetic programs generally follow an imperative programming paradigm where computation is driven procedurally.
Execution often starts at the top of a program and proceeds in sequence, instruction-by-instruction, jumping or branching as dictated by executed instructions \citep{brameier_linear_2007,mcdermott_genetic_2015}.
In contrast to the imperative programming paradigm, program execution in event-driven computing is directed primarily by signals (\textit{i.e.}, events), easing the design and development of programs that, much like biological organisms, must react on-the-fly to signals in the environment or from other agents. 
Is it possible to provide similarly useful abstractions to evolution in genetic programming? 

Different types of programs are more or less challenging to evolve depending on how they are represented and interpreted.  
By capturing the event-driven programming paradigm, SignalGP targets problem domains where agent-agent and agent-environment interactions are crucial, such as in robotics or distributed systems. 

In the following sections, we provide a broad overview of the event-driven paradigm, discussing it in the context of an existing event-driven software framework, cell signal transduction, and an evolutionary computation system for evolving robot controllers. 
Next, we discuss our implementation of SignalGP in detail. 
Then, we use SignalGP to demonstrate the value of capturing event-driven programming in GP with two test problems: an environment coordination problem and a distributed leader election problem. 
Finally, we conclude with planned extensions, including how SignalGP can be generalized beyond our linear GP implementation to other forms of GP. 